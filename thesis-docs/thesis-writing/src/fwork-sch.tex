\chapter{PhD story from now}



\section{Future work}
\label{sec:fwork}

The future work consists of exentions to the current framework across multiple
dimensions. I briefly mention them here:

\begin{description}
  \item[Semi-supervised version] The immediate and urgent task is to implement the semi-supervised
version of the model and evaluate it as soon as possible.
  \item[Improvements on runtime] Any broad-coverage semantic parser
should be able to be scaled to very large set of data. Albeit our current
framework is using either linear or bilinear models further investigation is
needed to improve the performance. The current candidate
for improving run-time of the model is to use factorization with less parameters, i.e.\ using vectors for modeling frames rather than matrices. This modification requires us to
revise our current factorization model and relax some of our model assumptions.
(not using frame-specific roles, for example.)
  \item[New learning objective] Our factorization model uses argument
  prediction as its learning objective, this idea can be extended to use
  paraphrases. The learning objective then, will be formulated as inducing
  similar semantic representation for pair of paraphrases. Another objective
  could be, using translation model. Given a pair of sentences in source
  language and its translation, the model should be able to induce similar
  semantic representation for the pair.
  \item[Machine reading and Q/A] Another dimension of work, is to see the
  application of our model in upstream NLP tasks such as question/answering
  (Q/A), machine reading/comprehension and relation extraction. All these tasks
  can benefit from powerful meaning representations. The major success depends
  on activating the potential ability of our framework to capture higher-order relations across predicates/frames.
  \item[Modality and Negation] Moreover, I would like to mainly
improve our framework to capture more sophisticated language usages. While our current framework is limited to simple examples of compositionality (ignoring noun and verb modifications), I would
like to push it for capturing negation and modality and go beyond propositional sentences. I have already gathered related
materials and datasets for this purpose and had insightful discussions with the
supervisor on this matter.
  \item[Reasoning] The upper limit
for this framework to learn how to perform logical reasoning should be improved
by substantial re-designing and introducing new features. One potential way is
to try to model logical connectives and quantifiers with tensors and vectors
which can be complicated, mostly due to this fact that this domain has been
started to be explored very recently.

\end{description}

\section{Schedule}
\label{sec:sch}

In this section I provide a coarse-grain time table for the sub-problems I want
to address for rest of the duration of studies.

\begin{description}
  \item[February 2015 - March 2015 ~~~~(2 months)]
  \hfill
  \begin{itemize}
    \item Working on semi-supervised version of the model
  \end{itemize}
  
  \item[April 2015 - September 2015  ~~~~(6 months)]\hfill 
  \begin{itemize}
    \item Working on frame induction using argument prediction objective
    \item Working on frame induction using paraphrase pairs (mono-lingual and
    cross-lingual)
  \end{itemize}
  
  
  \item[October 2015 - March 2016 ~~~~(6 months)]\hfill
	  \begin{itemize}
    	\item Applying achievements of the previous phases for Q/A and machine
    	comprehension, conditioned on success in previous levels
    	\item Working on shallow inference based on frames
	  \end{itemize}

  \item[April 2016 - September 2016 ~~~~(6 months)] \hfill 
	  \begin{itemize}
    	\item Continue working on inference based on frames
    	\item Working on modeling modality and negation
    
	  \end{itemize}

\item[October 2016 - March 2017 ~~~~(6 months)] \hfill
	  \begin{itemize}
    	\item Applying achievements of the previous phases for Q/A and machine
    	comprehension, conditioned on success in previous levels
    	\item Finalizing the gaps in the work and preparation for writing the
    	thesis
    
	  \end{itemize}

\item[April 2017 - September 2017 ~~~~(6 months)]\hfill
\begin{itemize}
    	\item Writing the thesis
    	\item Finalizing ready-to-publish work
    	    
	  \end{itemize}
\item[October 2017 - December 2017~~~~(3 months)] \hfill
\begin{itemize}
    	\item Submititting the thesis
    	\item Preparing for the defense
    	\item PhD defense
    	    
	  \end{itemize}
\end{description}


