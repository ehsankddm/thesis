\chapter{Future Work}
In ~\autoref{ch:text-kb} I proposed an idea to learn representation of entities
and relations encoded in a KB jointly with features extracted from a corpus.
This idea can be extended further by incorporating more features in form of
triples. Recently in \cite{Yao2012} and \cite{Riedel2010} topic
modelling and also extra features for entity types have been successfully
applied for relation discovery. We can also incorporate such features in to our
models, specially into \textit{Text+KB}. Any new type features can be expressed
as a new predicate, in a same formalization described in
~\autoref{sec:textkb-link}. For example, by using any of available topic
modeller methods we can assign a topic to each segment of text in our corpus and
relate each relation or NE in this part to its topic. For example if we have
assigned a topic of \textit{`politics`} for the part of a corpus which
contains this triple:
\begin{center}
 \textsc{ /government/position\_held/president (~Mir-Hossein Mousavi,~Iran)}
 \end{center}

then for any relation or NE occurred in this part we can introduce a
new feature:

\begin{center}
 \textsc{HAS\_TOPIC (~Mir-Hossein Mousavi,~Politics)}\\
 \textsc{HAS\_TOPIC (~Iran,~Politics)}\\
 \textsc{HAS\_TOPIC (~/government/position\_held/president,~Politics)}\\
 \end{center}

 This will probably lead to learn better features and therefore better
 discrimination between different NEs and relations. For topics, we can induce
 them by some methods like LDA in \cite{Blei2003} or use a corpus like Wikipedia
  which topics of each page is already provided.
  
  Likewise for the work in ~\autoref{ch:ent-link} more features from different
  resources can be embedded to the model. Some experiments has already been
  conducted to learn WordNet, FrameNet, OmegaWiki and Wikipedia word embeddings
  jointly which results will be reported later through future scientific
  publications. Also based on the results from the current work, we can predict
  cross-resource sense alignments and link entities across multiple resources.
  Our model can be used as a bootstrapping model which uses available
  cross-lingual or cross-resource multiple sense alignments to learn embeddings
  and then use same embeddings to predict new set of alignments. Currently we
  are in the phase of preparing evaluation dataset for this idea.
  
  One shortage of my work in  ~\autoref{ch:text-kb} is that the output of
  the model I used is not probabilistic. In \cite{Bordes2011} and
  \cite{Bordes2012} and in our model, each relation embedding induce a metric
  space which makes the similarity of different relations incomparable to each
  other. Because of this problem we can not compare it to some of other previous
  models such as \textit{Universal Schema} \cite{Riedel2013}. In
  \cite{Bordes2011} a trick based on kernel density estimation has been proposed and tested. 
  The slow performance of this trick makes it infeasible for large datasets.
  Currently I have implemented some alternative approaches for normalizing
  similarity scores across different relations but since the model is not
  designed to be probabilistic all the current tricks may not theoretically
  sound but can be practically useful to make it possible to compare our work to
  Riedel et al.'s universal schema.
  
  
