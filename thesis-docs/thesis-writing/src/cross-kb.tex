\chapter{Entity Linking Among Lexical Resources}
\label{ch:exp-ent-link}

Each KB or lexical resource contains limited amount of information.  

\section{Experimental Setup}
\label{sec:xkb-exp}

In this part we describe the methodology we followed to encode available information 
in two different lexical resources, WordNet and GermaNet, that makes it possible to link entities of the two different
resources and learn bi-lingual 
embeddings of word senses in German and English.  
The main idea is to relate two senses from two different resources using cross-lingual sense alignments.
This is an additional information which can play a role of bridge between two
different tasks, learning German embeddings and English embeddings, and can help to transfer knowledge from one to the another.
Using this new feature we make our WordNet-GermaNet dataset which contains three type of relations 
(1) WordNet relations 
(2) GermaNet relations
(3) Cross-lingual sense alignments between WordNet and GermaNet

First two types of relations are directly extracted from WordNet and GermaNet and for the 
cross-lingual relations we used Interlingual Index mappings between WordNet and GermaNet.
\\
Example of relations:
\begin{center}
WN-sense-A \hspace{0.3in}  WN-rel-1    \hspace{0.3in}  WN-sense-B\\
GN-sense-C \hspace{0.3in}  GN-rel-2    \hspace{0.3in}  GN-sense-D\\
WN-sense-A \hspace{0.3in}  ILI-rel-1-2 \hspace{0.3in}  WN-sense-B\\
\end{center}

Left and right entities are WordNet and GermaNet senses and relations are current semantical relations in each of lexicons such as:
 meronymy, holonymy and \ldots.
 
 We have created four different dataset, each divided to train, test and validation separated subsets. Our four datasets are:
 \begin{enumerate}
 \item Only WordNet triples (WN)
 \item Only GermaNet triples (GN)
 \item WordNet-GermaNet triples with one-direction cross-lingual alignments (WN-GN)
 \item WordNet-GermaNet with double-direction cross-lingual alignments (WN-GN DD)
 \end{enumerate}
 
 Dataset 3 includes both relations extracted from WordNet and GermaNet and also the mapping between senses.
 Dataset 4 is same as dataset 3 but since the models we will use are assuming all the relations are assymetric 
 we will try to encode the symmetry of cross-lingual alignments by reversing each of them and include the reverse in the dataset.
 Datasets 3 and 4 contains two different variants: the first variants contains only WordNet relations (test on WN) 
 in the held-out test dataset and the second variant contains only GermaNet triples (test on GN).
  In this way we can observe the direction of
  possibly transferring information from English to German or vice versa.
  
  For reducing the sparsity of data and boosting the learning runtime we 
  filtered out all the entities that appeared less than 3 times in our datasets.
   
  (version of wordnet, germanet, ILI and role of uby should be described here)  
 

\section{Evaluation}
\label{sec:ent-link-eval}

To show the effectiveness of joint learning of features from multiple knowledge bases we suggest 
two experiment setups. In the first schema we follow Bordes et al. ranking task. The goal of this task is
to show how well the information in knowledge bases can be preserved by the learned features. 
 On the other hand, the second
setup is investigating on this question that if the learned word embeddings from multiple resources
are able to improve the performance of monolingual embeddings in a standard NLP
task, here word-pair similarity or not.
In this setup we will look to contribution of the learned features in predicting similarity of words.

\subsection{Evaluation Using Reconstruction}
\label{ssec:ent-link-intrinsic}

Bordes et al. (Bordes2011) proposed a ranking task that for each triple $(e_{i} , r_{k}, e_{j} )$ in the data set,
 all the entities will be ranked as a candidate for being right entity of the triple 
 given the relation and the left entity. Depends on which one of the models is used, SE or SME-Bil, all the entities will be sorted
  based on their score regarding Equation~\eqref{eq:sim} or Equation~\ref{eq:dot} previously introduced in section~\ref{ssec:bordes}. 
  By keeping the statistics of difference between the predicted rank of $e_{j}$ and its true rank and also repeat the same process
  for left entities, we will be able to report the mean and median predicted rank of entities per relation and in total. Bordes et al.
   proposed to schema for calculating the average rank, micro averaging which emphasis on more frequent relations by
    weighted averaging with frequency of relations as weights and macro averaging which consider all the relations equally
    , either frequent or infrequent ones. The third statistic that we report following their work, r@100, is the ratio of number of times that 
    an entity is correctly among top 100 entities ranked and predicted for a triple to the number of occurances of this entity in the dataset.
    We applied SE and SME-Bil models on our created datasets and the ranking performance on each of them is presented in Table ~\ref{tbl:rank-tbl}.
	
\begin{table}[ht]
\caption{Intrinsic Evaluation (Ranking Score Performance)  }
\label{tbl:rank-tbl} % title name of the table
\centering % centering table

\begin{tabular}{l  c c c c c}
\hline\hline % inserting double-line
 Dataset &  \#relations & \#entities &  & Micro & Macro
\\ [0.5ex] 
\hline % inserts single-line

 & &   & lhs & 84.42 & 72.59 \\[-0.5ex]
  & &   & rhs & 84.04 & 72.38 \\[-0.5ex]
  & &   & mean & 1003.59 & 3739.85 \\[-0.5ex]
  & &   & median & 5.0 & 2213.37 \\[-0.5ex]
\raisebox{1.0ex}{GN SE} &  \raisebox{0.5ex}{16}& \raisebox{0.5ex}{64025}
&global& 84.23 & 72.49 \\[1ex]

 & &   & lhs & 79.06 & 58.58 \\[-0.5ex]
  & &   & rhs & 83.30 & 81.11 \\[-0.5ex]
  & &   & mean & 407.90 & 308.01 \\[-0.5ex]
  & &   & median & 10.0 & 54.18 \\[-0.5ex]
\raisebox{1.0ex}{GN SME-BIL} &  \raisebox{0.5ex}{16}& \raisebox{0.5ex}{64025}&global
&  81.18 & 69.85 \\[1ex]
\hline

 & &   & lhs & 91.90 & 89.47 \\[-0.5ex]
  & &   & rhs & 92.30 & 90.25 \\[-0.5ex]
  & &   & mean & 148.72 & 623.10 \\[-0.5ex]
  & &   & median & 5.0 & 4.69 \\[-0.5ex]
\raisebox{1.0ex}{WN SE} &  \raisebox{0.5ex}{23}& \raisebox{0.5ex}{148976}& global
& 92.10 & 89.86 \\[1ex]

& &   & lhs & 83.08 & 72.21 \\[-1ex]
  & &   & rhs & 85.2 &77.92 \\[-1ex]
  & &   & mean & 128.82 & 511.21 \\[-0.5ex]
  & &   & median & 10.0 & 26.63 \\[-0.5ex]
\raisebox{1.0ex}{WN SME-BIL} &  \raisebox{0.5ex}{23}& \raisebox{0.5ex}{148976}& global
& 84.14 & 75.57 \\[1ex]
\hline

 & &   & lhs & 90.82 & 89.14  \\[-0.5ex]
  & &   & rhs & 91.56 & 88.76 \\[-0.5ex]
  & &   & mean & 293.16 & 1356.30 \\[-0.5ex]
  & &   & median & 5.0 & 5.10 \\[-0.5ex]
\raisebox{1.0ex}{WN-GN SE (WN held out)} & \raisebox{0.5ex}{32}& \raisebox{0.5ex}{213002}& global
& 91.19 & 88.95 \\[1ex]

 & &   & lhs & 82.42 & 73.65 \\[-0.5ex]
  & &   & rhs & 83.40 & 73.44 \\[-0.5ex]
  & &   & mean & 124.85 & 331.82 \\[-0.5ex]
  & &   & median & 11.0 & 33.86 \\[-0.5ex]
\raisebox{1.0ex}{WN-GN SME-BIL(WN held out)} & \raisebox{0.5ex}{32}& \raisebox{0.5ex}{213002}& global
& 82.91 & 73.55 \\[1ex]

 \hline
 & &   & lhs & 81.82 & 70.56  \\[-0.5ex]
  & &   & rhs & 79.92 & 70.06 \\[-0.5ex]
  & &   & mean & 3031.44 & 15470.56 \\[-0.5ex]
  & &   & median & 7.0 & 10080.5 \\[-0.5ex]
\raisebox{1.0ex}{WN-GN SE (GN held out)} & \raisebox{0.5ex}{32}& \raisebox{0.5ex}{213002}& global
& 80.87 & 70.313 \\[1ex]

 & &   & lhs & 63.54 & 41.64 \\[-0.5ex]
  & &   & rhs & 64.78 & 70.32 \\[-0.5ex]
  & &   & mean & 984.79 & 1021.37 \\[-0.5ex]
  & &   & median & 40.0 & 428.90 \\[-0.5ex]
\raisebox{1.0ex}{WN-GN SME-BIL(GN held out)} & \raisebox{0.5ex}{32}& \raisebox{0.5ex}{213002}& global
& 64.16 & 55.98 \\[1ex]
 \hline
& &   & lhs & 57.72 & 38.063 \\[-0.5ex]
  &  &  & rhs & 60.72 & 63.617 \\[-0.5ex]
  & &   & mean & 932.49 & 719.47 \\[-0.5ex]
  & &   & median & 56.0 & 175.56 \\[-0.5ex]
\raisebox{1.0ex}{WordNet-GermaNet-DD (GN held out)} &  \raisebox{0.5ex}{32}& \raisebox{0.5ex}{213002}& global
& 59.22 & 50.84 \\[1ex]
 
& &   & lhs & 69.66 & 59.54 \\[-0.5ex]
  &  &  & rhs & 66.60 & 58.95 \\[-0.5ex]
  & &   & mean & 166.18 & 18.0 \\[-0.5ex]
  & &   & median & 466.91 & 55.41 \\[-0.5ex]
\raisebox{1.0ex}{WordNet-GermaNet-DD (WN held out)} & \raisebox{0.5ex}{32}& \raisebox{0.5ex}{213002}& global
& 68.13 & 59.25 \\[1ex]
 
% [1ex] adds vertical space
\hline % inserts single-line
\end{tabular}

\label{tab:PPer}
\end{table}
   

\begin{table}[ht]
\caption{Ranking Performance for Mapped Relations } % title name of the table
\centering % centering table
\begin{tabular}{l c c c c c c} % creating 10 columns
\hline\hline % inserting double-line
 Dataset & \#dimension & \#relations & \#entities &  & Micro(\%) & Macro(\%)
\\ [0.5ex] 
\hline % inserts single-line

 & & &  & lhs & 82.60 & 68.18 \\[-1ex]
  & & &  & rhs & 81.90 & 68.84 \\[-1ex]
\raisebox{1.5ex}{GermaNet} & \raisebox{0.5ex}{25}& \raisebox{0.5ex}{10}& \raisebox{0.5ex}{64025}&global
&  82.25 & 68.51 \\[1ex]

 & & &  & lhs & 83.50 & 83.17 \\[-1ex]
  & & &  & rhs & 84.22 & 83.64 \\[-1ex]
\raisebox{1.5ex}{WordNet} & \raisebox{0.5ex}{25}& \raisebox{0.5ex}{19}& \raisebox{0.5ex}{148976}& global
& 83.86 & 83.40 \\[1ex]

 & & &  & lhs & 78.70 & 82.60 \\[-1ex]
  & & &  & rhs & 79.56 & 83.06 \\[-1ex]
\raisebox{1.5ex}{WordNet-GermaNet (WN)} & \raisebox{0.5ex}{25}& \raisebox{0.5ex}{24}& \raisebox{0.5ex}{213002}& global
& 79.13 & 82.83 \\[1ex]

 & & &  & lhs & 69.66 & 59.54 \\[-1ex]
  & & &  & rhs & 66.60 & 58.95 \\[-1ex]
\raisebox{1.5ex}{WordNet-GermaNet (GN)} & \raisebox{0.5ex}{25}& \raisebox{0.5ex}{24}& \raisebox{0.5ex}{213002}& global
& 68.13 & 59.25 \\[1ex]


% [1ex] adds vertical space
\hline % inserts single-line
\end{tabular}
\label{tab:PPer}
\end{table}
    

\subsubsection{Evaluation on feature informativity}
\label{ssec:ent-link-extrinsic}
 \label{exp:word-similarity}
 We are interested to further analyze the effectiveness of learned embeddings to capture semantic features of words, therefor 
 we compare the embeddings learned a single resource or from multiple resources against human judgments.
 Five datasets of word-pair similarity are used to compare the correlation of predicted similairty of 
 pair of words against human judgments.
 [rubensteinGoodenough], [yangPowers], [millerCharles],[Szumlanski] and [finkelstein] are English datasets that we used 
 to meaure the correlation of similarities
 predicted by our embeddings and  embeddings induced by the other methods to human judgments. 
 For German, we use [this and that].
 The other embeddings which are used in our comparison are (Turian et al., HLBL and Klementiev et al.).
 To measure the similarity between any given wordpair $(w_1 , w_2)$ we find all vectors associated to different senses
 of the given words in our embedding dictionary and pick the pair of embeddings that maximize cosine similarity between two words. 
 We can motivate this by saying that for each word pair any of words works as a context for disambiguating the sense of the other word.
  
 Both Pearson and Spearman correlation of predicted and gold similarities 
 are calculated and is reported in table \ref{tbl:en-wp-sim} for English and \ref{tbl:de-wp-sim} for German.
 
 For English, we can see that \ldots.
 
 On the other hand in German \ldots. 
 
 \begin{table}[ht]
\caption{Word-pair Similarity Performance for English } % title name of the table
\label{tbl:en-wp-sim}
\centering % centering table
\tabcolsep=0.09cm

\begin{tabular}{cr c c c c c c c c c} 
\hline\hline % inserting double-line
 Dataset & & WN-SE  & WN-GN-SE & WN-SME-BIL &  WN-GN-SME-BIL & WN-GN-SME-BIL-DD &HLBL & Turian* & Klementiev*
\\ [0.5ex] 
\hline % inserts single-line
                                           &P&0.682&0.666&0.540&0.508&0.611&-0.115&0.233&-0.380 \\[-1ex]
\raisebox{1.5ex}{RubensteinGoodenough65}  &S&0.769&0.741&0.447&0.478&0.552&-.083&0.118&-0.398 \\[1ex]

                                     &  P &0.611&0.644&0.592&0.555&0.541&-0.363&0.150&-0.768 \\[-1ex]
\raisebox{1.5ex}{MillerCharles30}  &  S & 0.720&0.648&0.564&0.561&0.468&-.450&-0.198&-0.522 \\[1ex]

                                   &  P &0.179&0.206&0.272&0.208&0.193&0.236&0.246&0.032 \\[-1ex]
\raisebox{1.5ex}{Finkelstein353}  &  S &0.087&0.146&0.240&0.196&0.162&0.204&0.223&0.044 \\[1ex]

                                  &  P &-0.145&0.032&0.010&0.043&0.048&-0.228&0.014&0.0001 \\[-1ex]
\raisebox{1.5ex}{Szumlanski122}  &  S & -0.159&0.034&0.035&0.037&0.041&-0.263&0.023&-0.014 \\[1ex]

                                  &  P & 0.729&0.682&0.597&0.767&0.819&0.237&0.199&0.454 \\[-1ex]
\raisebox{1.5ex}{YangPowers130}  &  S & 0.829&0.853&0.483&0.793&0.836&0.024&0.097& 0.634 \\[1ex]

\hline % inserts single-line
     
          
 \hline % inserts single-line
\end{tabular}

\label{tab:en-wp-sim}
\end{table}      


\begin{table}[ht]
\caption{Word-pair Similarity Performance for German } % title name of the table
\label{tbl:de-wp-sim}
\centering  % centering table
\tabcolsep=0.09cm

\begin{tabular}{cr c c c c c c c} % creating 10 columns
\hline\hline % inserting double-line
 Dataset & & GN-SE  & WN-GN-SE & GN-SME-BIL &  WN-GN-SME-BIL & WN-GN-SME-BIL-DD & Klementiev*
\\ [0.5ex] 
\hline % inserts single-line
                                 &  P & -0.022  & 0.112 & 0.058 & 0.103&0.203 &0.118 \\[-1ex]
\raisebox{1.5ex}{wortpaare222}  &  S & -0.100 & 0.225 & 0.230 & 0.091 & 0.195 &0.153 \\[1ex]

                                  &  P & 0.865 & 0.984 & -0.443 & 0.671 & 0.656 & -0.887 \\[-1ex]
\raisebox{1.5ex}{wortpaare30}    &  S & 1.0 & 1.0 & -0.500 & 0.682 & 0.686 & -1.0 \\[1ex]

                                  &  P & -0.089  & 0.045 & 0.163 & 0.300& 0.256 &0.168 \\[-1ex]
\raisebox{1.5ex}{wortpaare350}  &  S & -0.158 & -0.017  &  0.135 & 0.295 & 0.231 &0.117 \\[1ex]

                                &  P & 0.800  & 0.558 & -0.572 & 0.607 & 0.480 & 0.233 \\[-1ex]
\raisebox{1.5ex}{wortpaare65}  &  S & 0.800 & 0.800 & -0.8 & 0.588 & 0.439 &0.200 \\[1ex]


\hline % inserts single-line
     
          
\hline % inserts single-line
\end{tabular}


\end{table}      
          
            
As we see in the table \ref{tab:en-wp-sim} in two datasets the performance of learned embeddings from bi-lingual resources
are slightly worse but comparable to the mono-lingual embeddings and in the other two datasets one can observe a significant 
increase of performance of bi-lingual resources over monolingual resources.      


More analysis on why some dataset is good and some is not good. 