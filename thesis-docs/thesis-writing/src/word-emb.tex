\chapter{Word Embeddings}
\label{ch:word-embeddings}

\section{Neural Language Models}
\label{ssec:lang-model}

In \cite{Collobert2008a}, Weston and Collobert use a non-probabilistic and
discriminative model to jointly learn word embeddings and a language model that
can separate plausible n-grams from noisy ones. For each word in a n-gram, they
combine the word embeddings and use it as positive example. They put noise in
the n-gram to make negative examples and then train a neural network to learn to
classify postive labels from negative ones. The parameters of neural network
(neural language model) and word embedding values will be learned jointly by an
optimization method called \emph{Stochastic Gradient Descent} \cite{Bottou2010}.

A hierarchical dsitributed language model (HLBL) proposed by Mnih and
Hinton in \cite{Mnih2009} is another influential work on word embeddings. In
this model a probabilistic linear neural network(LBL) will be trained to 
combine word embeddings in first $n-1$ words of a n-gram to predict the $n_th$ word.

Weston-Collobert model and HLBL by Mnih and Hinton are evaluated in
\cite{Turian2010b} in two NLP tasks: chunking and named entity recognition. With
using word embeddings from these models combined with hand-crafted features, the
performance of both tasks are shown to be improved.

\section{Learning Word Embeddings from Lexical Resources}
\label{ssec:structured-word-embedding}



