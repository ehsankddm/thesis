\chapter{Introduction}
Nowadays searching through Internet is the first step we take if we want to get answer to our question.
We convert our questions to sequences of keywords and search engines are trying 
to lead us to web pages where might contain the answer to our questions, they return us a set of related
documents which are sorted by their popularity and similarity of their text to our query.
In this way, users themselves are responsible to find the desired knowledge 
from documents. The next generation search engines should go further than current 
approaches in understanding the meaning of a query and the underlying semantics of documents on 
the web. The next natural improvement is to return a piece of information which directly approaches to 
answer the user question. For this reason, the elements of a query, concepts or entities and 
their relations should be identified and documents ( which might have the answer) should be mapped to the 
same space of entities and relations in order to find the the desired information in the question.


Relation Extraction is the task of detecting and classifying semantic relationship 
between named entities (NE). The goal of this task is to find a triple
of binary relations and their arguments \cite{Androutsopoulos2009}. For instance, we want to induce a
relation like \emph{bornIn} with its arguments which could be for example like this:\emph{ (
bornIn, Beata Nyari, Budapest )}. Applications of such task is numerous
in natural language processing; Question/Answering, machine translation and
text summarization are systems that benefit from relation extraction \cite{Androutsopoulos2009}.

In this task, we are trying to have a model to find paraphrases which means
that we are interested to have all the similar semantically similar relations under
one umbrella. So it is desired to have all different surface realizations of one
relation like \emph{isGivenBirth, isBorn, isFrom} in a same set, namely \emph{bornIn}.

From a classic method, DIRT \cite{Lin2001}, to very famous frameworks ,
TextRunner\cite{Bankoa} or distant supervision\cite{Mintz2009}, and more recent works e.g. PATTY\cite{Nakashole2012a},
all are examples of several different family of approaches. These methods could
be categorized from different perspectives (1) amount of annotated data they
need (2) if they can only handle a predefined enumeration of entities and re-
lations or are open to any number of relations (3) organization of semantic
interpretation and (4) underlying family of methods they use.

In this proposal, we suggest a new method which learns informative features from
data using deep learning methods and then use the learnt features to develop a
relation extraction system in two schema (1) Weakly supervised and (2)
Unsupervised. 

In next chapters we review recent or influential related works and finally
we will describe our new model as an improvement to current approaches.
