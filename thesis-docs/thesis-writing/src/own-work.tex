\chapter{PhD story so far}
Nowadays searching through Internet is the first step we take if we want to
answer our question.
We convert our questions to sequences of keywords and search engines are trying 
to lead us to web pages where might contain the answer to our questions, they return us a set of related
documents which are sorted by their popularity and similarity of their text to our query.
In this way, users themselves are responsible to find the desired knowledge 
from documents. The next generation search engines should go further than current 
approaches in understanding the meaning of a query and the underlying semantics of documents on 
the web. The next natural improvement is to return a piece of information which directly approaches to 
answer the user question. For this reason, the elements of a query, concepts or entities and 
their relations should be identified and documents ( which might have the answer) should be mapped to the 
same space of entities and relations in order to find the the desired knowledge
asked in the question. For the sake of achieving to this level of intelligence,
we need a broad-coverage and open domain \emph{Semantic Parser} that helps our
software to deal with complexity of language. My main activity during my PhD
studies is to find solution to subset of problems that hinder us in obtaining
such semantic parsers.

\section{Semantic parsing}
\label{sec:usp-sem-parse}

Semantic parsing is finding a mapping from a sentence (surface structure) to its
formal meaning representation (hidden structure).
We aim to represent a natural language text with first-order logic. One can derive
a semantic parse of a sentence by starting from a lexicon of atomic formulas and combining each fragment to build a composition of formulas combined with quantifiers and logical connectives. In \citep{Poon2009} the lexicon will be induced from a raw corpus. It is in contrast to traditional means of semantic parsing with manually
  produced lexicons.
  
  The main challenge in unsupervised semantic parsing is that for a single semantic representation there could be
  several syntactic and lexical realizations (different surface
  representations). For example, all of the sentences below has a same semantic
  representation:
   \begin{itemize}
     \item Microsoft buys Skype
     \item Microsoft acquires the VoIP company Skype
     \item Skype is acquired by Microsoft Corporation
     \item The Redmond software giant buys Skype
     \item Microsoft’s purchase of Skype,\ldots
   \end{itemize}  
  
   A simple lexicon to represent all of the examples above is:
  
   $$ BUY(n_1)$$
   $$ \lambda x_2.BUYER(n_1 , x_2) \; \;  \lambda x_3.BOUGHT(n_1 , x_3) $$
   $$ MICROSOFT(n_2) \; \; SKYPE(n_3)$$
   
   Having a corpus of surface structures in natural language, one can aim to
   induce such a lexicon and will also extract a formal representation for each sentence.
   %In the next section we will review the necessary steps toward this goal.
   Although, since a series of tasks that interest me are usually dealing with
   factual or situated examples of language (e.g.\ relation extraction,
   question answering) with limited compositionality, we can choose a framework,
   namely \textit{Frame theory}, which gives us more flexibility than logical
   representations but with trading-off the expressiveness power.
   
   \textbf{Frames} are conceptual structures that describe an event or situation
   with specifying each participant and its role \cite{Baker1998}. From my
   perspective, frames are generalization of verb  senses (also nominal and
   prepositional phrases in some cases) with additional information on number
   of semantic arguments and restrictions on their types (generalization of
   selectional preferences).
   
   \textit{Frame parsing} is the task of mapping the
syntactic analysis of a sentence to predicate-argument structure and label each
argument with set of semantic roles. Predicates should also map to additional
layer of abstraction which groups them in to frames.

In the examples below, frames are identified (for clarity one frame per
sentence but there is no limit on it) and its arguments are marked for each
sentence (argument identification) and each argument is labeled with its semantic role (argument classification). 

\begin{enumerate}
  
    \item \lbrack Anakata\rbrack_{author}  \lbrack {writes}
    \rbrack_{Text-creation} his \lbrack
    emails\rbrack_{text} with \lbrack
    Emacs\rbrack_{instrument}.
    \item  \lbrack Emacs\rbrack_{instrument} is
    used to \lbrack {compose}
    \rbrack_{Text-creation}  \lbrack
    emails\rbrack_{text} by \lbrack
    Anakata\rbrack_{author}.
  \end{enumerate}

On beginning, I had the impression that syntax can be quite powerful and
can provide enough information for semantic analysis. But as we can see, these
examples can show us the challenge of mapping from syntax to semantics.
while \emph{Emacs} have different syntactic roles in above examples, its
semantic role does not change and additionally, the predicates with one frame,
\emph{text creation}, has two different lexical realizations: \emph{writing} and
\emph{composing}.
This shows that there is no one-to-one mapping between surface syntax analysis and the semantic roles while syntax remains as a strong signal. The mapping
from syntactic function to semantic roles of a verbal argument is commonly 
called \textbf{linking} in the literature.
\citep{Grenager2006,Lang2011a,Lang2011b}
   
\subsection{Unsupervised semantic parsing}
\label{ssec:usp-sem-parse}
In order to automatically parse sentences to their semantic-frame
representations, one can train a statistical model on available
manually annotated resources.
While this approach can be quite effective given a thorough and richly-annotated
resource but because of major flaws highlighted below, unsupervised methods
(annotation-free models) can be used as a replacement or complementary medium
for approaching the task.

\begin{description}
\item[New domains need new annotations] \hfil \\ Although incorporating
annotated corpus and supervised approaches can achive high accuracy for in-domain sentences 
\cite{Punyakanok2008,Marquez2008}, they have been
shown to perform poorly on open-domain texts \cite{Pradhan2008}.
\item[Difficulty of multi-lingual annotation] \hfil \\ The difficulty
of prdoucing high-quality annotations for a large number of languages from
perspective of time and budget is another reason to move toward unsupervised
models where little or no human effort is needed.
\item[Granularity of abstraction] \hfil \\ In FrameNet, predicates are assigned
to a same semantic frame if they usually co-occure in a same situation and roles are
frame-specific. In contrast to FrameNet, PropBank follow a different strategy
where there is no structured relation between predicates, namely each frame is
only being assigned to one predicate.
Also roles ar meant to be specific to each verb and formally do not share a
structure together. Additionally, opposite verbs (antonyms) are assigned to a
same frame. Depends on a desired NLP task, one should find a trade-of between
these levels of granularity.

\end{description}

Above facts hint us to aim for unsupervised learning of semantic-frames. Looking
back to set of work on this direction and with help of my supervisor I
understood that that succesful models are dominated by idea of using
generative models. Generative models are probabilistic models which estimate the joint distribution of
observable variables (surface forms) and hidden variables (meaning
representation). The so-called generative story of the model describe the
interaction between hidden and observable variables and should be designed for
specific scenario by researchers. Because of difficulty of estimation of joint
probabilities, generative stories tend to use as few number as possible
variables. Therefore, a small set of features (observables) but with very strong
signals (informative features) are usually used. In the specific case
of unsupervised frame learning, the features are usually strong syntactic
signals. This can limit models and it does in many ways such as: 1) the model
over-relies on syntax and can not easily abstract away the syntax to capture
semantics, 2) Adapting model from one language to another one needs a
substantial re-designing of features and language specific knowledge. These are
the main flaws in previous work that my supervisor and I were trying to address
in last 14 months. In the next section I will briefly describe our solution for
such flaws and demonstrate our results.

\subsection{Feature-rich unsupervised learning of frames and semantic roles}
\label{ssec:feat-usp-sem-parse}

The materials of this section are based on our joint work with my supervisor and
briefly summarize our solution for effective unsupervised learning of meaning
representations. There are two recent scientific papers demonstrating our work
which both have passed the review phase and waiting for the final decision. The
first paper, \emph{``Unsupervised Induction of Semantic Roles within a
Reconstruction-Error Minimization Framework''} is submitted to NAACL
2015\footnote{The North American Chapter of the Association for Computational Linguistics} as a long paper and the second paper, 
\emph{``Inducing Semantic Representation from Text by Jointly Predicting and
Factorizing Relations''} is submitted to the workshop track of ICLR
2015\footnote{International Conference of Learning Representation}. The preprint
version of both papers are available online in \emph{arXiv}.


In our work, we propose a framework which address the main problem of previous
similar models, namely using few strong syntactic features. In our framework,
instead of using few language-dependent features, we let our model
benefit from huge space of extraced features, as it is common in supervised
methods. The difference is that, here we do not have access to labeled data so
unlike supervised methods we can not benefit from that which makes the problem
hard. 


\subsubsection{Model description}
\label{sssec:model}
Our model consists of two components: a log-linear feature rich semantic role labeler and
a tensor-factorization model which captures interaction between semantic roles and argument fillers. 

For the first component, we can use any existing approach for semantic role
labeling as long as the posterior of frames and roles can be computed
effectively. The input data to this component is the feature-vector
representation of a sentence which arguments are identified and this component
is responsible to assing a frame to the predicates and roles to arguments.
 
consider this sentence where arguments are shown by italic and predicate by
bold font:

\begin{center}
 \textit{Anakata} \textbf{writes} his \textit{emails} with  \textit{Emacs}.
\end{center}

the goal of the first component is to take the input sentence as feature-vector
representation and annotates it with cluster-id of induced frames and roles:

\begin{center}
\lbrack Anakata\rbrack_{1}  \lbrack {writes}
    \rbrack_{F23} his \lbrack
    emails\rbrack_{3} with \lbrack
    Emacs\rbrack_{10}.
\end{center}

How can we train the semantic role labeler without supervision? This is the
responsibility of the second component to provide a signal for the semantic role
labeler to find the best and menaingful configuration of frames and
roles. The second component ovjective is prediction of argument tuples based
on roles and the predicate given by the first model. The assumption is that, if
the signal (role assignments) of the first model is good enough then the second
model is able to perform well, otherwise if the second model is unable to
predict missing arguments, it means information from the first model has bad
quality. In conclusion, training these two components jointly should lead to the
model that role assignments are mostly corresponding to roles and frames
proposed by linguists. 

For example, the second component should be able to predict the word
\emph{emails} in the example if other arguments and all of roles are given:

\begin{center}
\lbrack Anakata\rbrack_{1}  \lbrack {writes}
    \rbrack_{F23} his \lbrack
    \st{emails}\rbrack_{3} with \lbrack
    Emacs\rbrack_{10}.
\end{center}

For each argument we learn a vector which encodes semanitc knowledge of this
specific argument. Moreover, we learn a matrix (transformation) for each
frame-role pair. Furthermore, the model calculates the similarity score between
the embedding of an argument and the transformed embeddings of the other argument candidates. This score can
be used for training the model. Why do we think that the second component can
provide meaningful scores? Mainly because predicate-argument representation is
capturing semantic core of the sentence(e.g.\ a situation, a relation or an
event).
Thus, these representations, rather than surface linguistic details like argument order or syntactic functions, 
should be crucial for narrowing down the set of potential candidates for
argument tuples.

In our papers we refer to our first component as: \emph{encoding
model}, and the second component as: \emph{reconstruction model} to point the
link to the idea of autoencoders in artificial neural networks. Although, our model, unlike
those, is not reconstructing the whole input space (feature representation) and
also is using two different family of models as encoding and decoding
components.
While our first model is set of independent log-linear classifiers, our reconstruction
component is a tensor factorization model which factorizes over frames and roles
coupled by arguments. 

For the details of the model and our inference algorithm, I refer the readers to
our papers. 

As like previous models, our framework should be evaluated experimentally. We
used the standard setup in previous models~\citep{Lang10} by evaluating our
model against the CoNLL 2008 shared task data~\citep{conll08} for English and
the CoNLL 2009 shared task for German.

The English dataset contains a version of Penn Tree-bank and PropBank which
automatic dependency analyses are additionally included. This dataset is divided
to three sgements of training, development and test. We took the largest
segment namely the training subset for producing role labelings and comparing it
to gold data. The other two segments of the data, development and testing, were
used to choose a set of paramateres that yields the best
labeling of roles compared to the annotation. 

As in most previous work on unsupervised SRL, we evaluate our model using clustering metrics: purity, collocation and their harmonic mean F1. 
\emph{Purity} (PU) measures the average number of arguments with the same gold
role label in each cluster, \emph{collocation} (CO) measures to what extent a specific gold role is represented by a single cluster. 
F1 is their harmonic mean.
For actual formula to calculate purity and collocation we refer to our long
paper.

For the semantic role labeling (encoding) component,
we relied on 14 feature patterns used for argument labeling in one of the
popular supervised role labelers~\citep{johansson2008}, which resulted in a
quite large feature space (49,474 feature instantiations for our English dataset).

\subsubsection{Experimental evaluation}
\label{sssec:eval}
Following ~\citep{Lang10}, we use a baseline ({\em SyntF}) which simply
clusters predicate arguments according to the dependency relation to their head.
A separate cluster is allocated for each of 20 
most frequent relations in the dataset and an additional cluster is used for all
other relations.
As observed in the previous work~\citep{Lang11a}, this is a hard baseline to beat. %, especially when, as in this case, the gold standard syntax is used.

We also compare against previous approaches:
the latent logistic classification model \citep{Lang10}  (labeled {\em LLogistic}), 
the  agglomerative clustering  method \citep{Lang11a} ({\em Agglom}),  the graph partitioning approach \citep{Lang11b} ({\em GraphPart}),
the global role ordering model~\citep{garg2012} ({\em RoleOrdering}). We also report results of an  improved version of {\em Agglom},
recently reported by~\citet{Lang14} ({\em Agglom+}).
The strongest previous model is {\em Bayes}: {\em Bayes} is the most accurate (`coupled') version of  
the Bayesian model of~\citet{TitovKlementEacl12},
estimated from the CoNLL data without relying on any external data.

\begin{table}[t]
\caption{Results on English (PropBank / CoNLL 2008).}
\begin{center}

\begin{tabular} {l c c c}  %{l*{5}{c}r}

\label{tab:en}
                & PU & CO & F1 \\
\hline
Reconst(Our model)         & 79.7 & 86.2 & {\bf 82.8}   \\
Bayes           & 89.3 & 76.6 & 82.5   \\
Agglom+          &  87.9 & 75.6 & 81.3 \\
RoleOrdering    & 83.5 & 78.5 & 80.9 \\
Agglom        & 88.7 & 73.0 & 80.1   \\
GraphPart		  &	88.6 & 70.7 & 78.6   \\
%Linking$^{*}$          &  74.5 & 83.1 & 78.6 \\
LLogistic         & 79.5 & 76.5 & 78.0   \\

\hline
SyntF		      & 81.6 & 77.5 & 79.5   \\
\end{tabular}

\end{center}
\end{table}



For German, we replicate the experimental set-up previously
used by Titov and Klementiev~\citet{TitovAcl2012}. As for English,
we report results of the syntactic baseline ({\em SyntF}).  
The results for all approaches are presented in Table~2.
We compare against {\em Bayes (De)} -- the {\em Bayes} model~
with argument signatures specialized for German (as reported in Titov and
Klementiev~\citet{TitovAcl2012}).
We also consider the original version of the {\em Bayes} model (denoted as {\em Bayes (En)}).
 
\begin{table}
\caption{Results on German (SALSA / CoNLL 2009).}
\begin{center}
\begin{tabular}{l*{5}{c}r}
                & PU & CO & F1 \\
\hline
Reconst(Our model)         & 76.3 & 87.0 & \textbf{81.3}   \\
Bayes (De)         		   & 86.8 & 75.7 & 80.9   \\
Bayes (En) 				   & 80.6 & 76.0 & 78.3 \\
\hline
SyntF		      & 83.1 & 79.3 & 81.2  \\
\end{tabular}
\end{center}

\end{table}

For German, we replicate the experimental set-up previously
used by Titov and Klementiev~\citet{TitovAcl2012}. As for English,
we report results of the syntactic baseline ({\em SyntF}).  
The results for all approaches are presented in Table~2.
We compare against {\em Bayes (De)} -- the {\em Bayes} model~
with argument signatures specialized for German (as reported in Titov and
Klementiev~\citet{TitovAcl2012}).
We also consider the original version of the {\em Bayes} model (denoted as {\em Bayes (En)}). 

The overall picture for German closely resembles the one for English. Our method achieves results comparable to
the best method evaluated in this setting. 
Importantly, parameters and features of our model for German and English are identical. 
On the contrary, by comparing  {\em Bayes (En)} with {\em Bayes (De)},  one can see that specialization of argument
signatures was crucial for the Bayesian model.  Also, similarly to English, our method induces less fine-grain sets of semantic roles but achieves
much higher collocation scores. This can be seen as an advantage since our roles
are linguistically justifiable.




