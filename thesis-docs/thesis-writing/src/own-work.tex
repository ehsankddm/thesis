\chapter{Story so far}
Nowadays searching through Internet is the first step we take if we want to
answer our question.
We convert our questions to sequences of keywords and search engines are trying 
to lead us to web pages where might contain the answer to our questions, they return us a set of related
documents which are sorted by their popularity and similarity of their text to our query.
In this way, users themselves are responsible to find the desired knowledge 
from documents. The next generation search engines should go further than current 
approaches in understanding the meaning of a query and the underlying semantics of documents on 
the web. The next natural improvement is to return a piece of information which directly approaches to 
answer the user question. For this reason, the elements of a query, concepts or entities and 
their relations should be identified and documents ( which might have the answer) should be mapped to the 
same space of entities and relations in order to find the the desired information in the question.

\section{Semantic Parsing}
\label{sec:usp-sem-parse}

Semantic parsing is finding a mapping from a sentence (surface structure) to its
formal meaning representation (hidden structure).
We aim to represent a natural language text with first-order logic. One can derive
a semantic parse of a sentence by starting from a lexicon of atomic formulas and combining each fragment to build a composition of formulas combined with quantifiers and logical connectives. In \citep{Poon2009} the lexicon will be induced from a raw corpus. It is in contrast to traditional means of semantic parsing with manually
  produced lexicons.
  
  The main challenge in unsupervised semantic parsing is that for a single semantic representation there could be
  several syntactic and lexical realizations (different surface
  representations). For example, all of the sentences below has a same semantic
  representation:
   \begin{itemize}
     \item Microsoft buys Skype
     \item Microsoft acquires the VoIP company Skype
     \item Skype is acquired by Microsoft Corporation
     \item The Redmond software giant buys Skype
     \item Microsoft’s purchase of Skype,\ldots
   \end{itemize}  
  
   A simple lexicon to represent all of the examples above is:
  
   $$ BUY(n_1)$$
   $$ \lambda x_2.BUYER(n_1 , x_2) \; \;  \lambda x_3.BOUGHT(n_1 , x_3) $$
   $$ MICROSOFT(n_2) \; \; SKYPE(n_3)$$
   
   Having a corpus of surface structures in natural language, one can aim to
   induce such a lexicon and will also extract a formal representation for each sentence.
   %In the next section we will review the necessary steps toward this goal.
   Although, since a series of tasks that interest me are usually dealing with
   factual or situated examples of language (e.g.\ relation extraction,
   question answering) with limited compositionality, we can choose a framework,
   namely \textit{Frame theory}, which gives us more flexibility than logical
   representations but with trading-off the expressiveness power.
   
   \textbf{Frames} are conceptual structures that describe an event or situation
   with specifying each participant and its role \cite{Baker1998}. From my
   perspective, frames are generalization of verb  senses (also nominal and
   prepositional phrases in some cases) with additional information on number
   of semantic arguments and restrictions on their types (generalization of
   selectional preferences).
   
   \textit{Frame parsing} is the task of mapping the
syntactic analysis of a sentence to predicate-argument structure and label each
argument with set of semantic roles. Predicates should also map to additional
layer of abstraction which groups them in to frames.

In the examples below, frames are identified (for clarity one frame per
sentence but there is no limit on it) and its arguments are marked for each
sentence (argument identification) and each argument is labeled with its semantic role (argument classification). 

\begin{enumerate}
  
    \item \lbrack Anakata\rbrack_{author}  \lbrack {writes}
    \rbrack_{Text-creation} his \lbrack
    emails\rbrack_{text} with \lbrack
    Emacs\rbrack_{instrument}.
    \item  \lbrack Emacs\rbrack_{instrument} is
    used to \lbrack {compose}
    \rbrack_{Text-creation}  \lbrack
    emails\rbrack_{text} by \lbrack
    Anakata\rbrack_{author}.
  \end{enumerate}

These examples aim to show the challenge of mapping from syntax to semantics.
while \emph{Emacs} have different syntactic roles in above examples, its
semantic role doesn't change and additionally, the predicates with one frame,
\emph{text creation}, has two different lexical realizations: \emph{writing} and
\emph{composing}.
This shows that there is no one-to-one mapping between surface syntax analysis and the semantic roles while syntax remains as a strong signal. The mapping
from syntactic function to semantic roles of a verbal argument is commonly 
called \textbf{linking} in the literature.
\cite{Grenager2006,Lang2011a,Lang2011b}
   
In order to automatically parse sentences to their semantic-frame
representations, one can train a statistical model on available annotations.
While this approach can be quite effective given a thorough and richly-annotated
resource but because of major flaws highlighted below, unsupervised methods
(annotation-free models) can be used as a replacement or complementary medium
for approaching the task.

\begin{description}
\item[New domains need new annotation] \hfil \\ Although incorporating
annotated corpus and supervised approaches can achive high accuracy for in-domain sentences 
\cite{Punyakanok2008,Marquez2008}, they have been
shown to perform poorly on open-domain texts \cite{Pradhan2008}.
\item[Difficulty of multi-lingual annotation] \hfil \\ The difficulty
of prdoucing high-quality annotations for a large number of languages from
perspective of time and budget is another reason to move toward unsupervised
models
\item[Granularity of abstraction] \hfil \\ In FrameNet, predicates are put to a
same semantic frame if they usually co-occure in a same situation and roles are
frame-specific. In contrast to FrameNet, PropBank follow a different strategy
where there is no structured relation between predicates. Also roles ar meant to
be specific to each verb and formally do not share a structure together. Depends
on desired NLP task, one should find a trade-of between these levels.

\end{description}

